
\section{试题研究}
\label{sec:problem-study}

\subsection{江苏卷数列题目结论的推广}
\label{sec:jiangshu-series-extend}

\textbf{作者}: 成都-zhcosin

江苏卷的$P(k)$数列题目的结论可以推广为,设$r$和$s$是两个不同的正整数,且都大于1,若某数列既是$P(r)$数列又是$P(s)$数列,则它一定是等差数列。

\begin{proof}[\textbf{证明}]
  设数列是$P(m)$数列,则有递推公式
  \[ a_{n-m}+a_{n-m+1}+\cdots+a_{n-1}-2ma_n+a_{n+1}+\cdots+a_{n-m+1}+a_{n+m}=0 \]
  这是齐次线性递推数列,其特征方程是
  \[ 1+x+\cdots+x^{m-1}-2mx^m+x^{m+1}+\cdots+x^{2m}=0 \]
  为了得到它的通项表示,讨论它的根的情况,记左边的多项式为$f(x)$,则$f(1)=f'(1)=0$,所以$x=1$是它的二重根。设它的其它复数根分别为$t_i(i=2,3,\cdots,q)$,则通项可表为
  \[ a_n=c_0+c_1 n + \sum_{i=2}^q p_i(n)t_i^n \]
  其中的系数$c_0,c_1$为复常数,$p_i(n)$是一些关于$n$的多项式。

  如果数列同时是$P(r)$数列和$P(s)$数列,那么数列的通项将有两种上述表达式,要证明它是等差数列,则只需要证明通项中那些多项式全是零多项式就行了,于是只要证明如下的引理:
  \begin{lemma}
    设$t_1,t_2,\cdots,t_m$是$m$个非零且互不相等的复常数,而$p_i(x)(i=1,2,\cdots,m)$一些关于$x$的多项式,如果等式$p_1(n)t_1^n+p_2(n)t_2^n+\cdots+p_m(n)t_m^n=0$对一切正整数$n$都成立,则$p_i(x)(i=1,2,\ldots,n)$全是零多项式。
  \end{lemma}
  证明如下:

  由条件知有如下等式
  \begin{equation*}
    \begin{pmatrix}
      t_1 & t_2 & \cdots & t_m \\
      t_1^2 & t_2^2 & \cdots & t_{m^2} \\
      \cdots \\
      t_1^r & t_2^r & \cdots & t_m^r
    \end{pmatrix}
    \cdot
    \begin{pmatrix}
      p_1(1) & p_1(2) & \cdots & p_1(r) \\
      p_2(1) & p_2(2) & \cdots & p_2(r) \\
      \cdots \\
      p_m(1) & p_m(2) & \cdots & p_m(r)
    \end{pmatrix}
    = O
  \end{equation*}
  上式中$r$可以一直写到任意正整数,由上式便知,$p_i(k)(i=1,2,\ldots,m)$是下面齐次线性方程的解
  \begin{equation*}
    \left\{
        \begin{array}{l}
         p_1(k)t_1+p_2(k)t_2+\cdots+p_m(k)t_m=0 \\
         p_1(k)t_1^2+p_2(k)t_2^2+\cdots+p_m(k)t_m^2=0 \\
          \cdots \\
         p_1(k)t_1^r+p_2(k)t_2^r+\cdots+p_m(k)t_m^r=0 \\
        \end{array}
        \right.
  \end{equation*}
  其中$k=1,2,\ldots$,取$r=m$时,方程个数与未知数个数相同,由范德蒙行列式可得
  \begin{equation*}
    \begin{vmatrix}
      t_1 & t_2 & \cdots & t_m \\
      t_1^2 & t_2^2 & \cdots & t_{m^2} \\
      \cdots \\
      t_1^m & t_2^m & \cdots & t_m^m
    \end{vmatrix}
    = t_1t_2\cdots t_m \prod_{1 \leqslant i < j \leqslant m}(t_i-t_j) \neq 0
  \end{equation*}
  因此方程的系数行列式非零,故上述方程只有零解,即
  \[ p_1(k)=0, p_2(k)=0, \cdots, p_m(k)=0 \]
  而$k$又是任意的正整数,所以$p_i(x)(i=1,2,\ldots,m)$只能全是零多项式。于是得证.
\end{proof}


%%% Local Variables:
%%% mode: latex
%%% TeX-master: t
%%% End:
