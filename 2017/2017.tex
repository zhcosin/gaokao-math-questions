
\documentclass{ctexart}
\input{use-packages}

% 自定义环境
\newcounter{example}[section]
\renewcommand{\theexample}{\thesection.\arabic{example}}

\newenvironment{example}[1][]{\refstepcounter{example} \textbf{例 \theexample  \ #1} \hspace{0.5em}}{\hspace{\stretch{1}} \rule{1ex}{1ex}}

\newcounter{exercise}[section]
\renewcommand{\theexercise}{\thesection.\arabic{exercise}}

\newenvironment{exercise}[1][]{\refstepcounter{exercise} \textbf{试题 \theexercise  \ #1} \hspace{0.5em}}{}

\newtheorem{definition}{定义}[section]
\newtheorem{property}{性质}[section]
\newtheorem{theorem}{定理}[section]
\newtheorem{inference}{推论}[section]
\newtheorem{axiom}{公理}[section]
\newtheorem{lemma}{引理}[section]
\newtheorem{principle}{原理}[section]
%\newtheorem{exercise}{题目}[section]
\newtheorem{topic}{问题}[section]
\newtheorem{statement}{命题}[section]
% \newtheorem{example}{例}[section]

% 使公式编号与章节关联,命令由 amsmath 宏包提供
\numberwithin{equation}{section}

% 配合 \autoref 命令使引用不只引用编号,也能引用环境命名,如: 定理 3.2.5,来自 hyperref 宏包。
%\newcommand\equationautorefname{式}
%\newcommand\footnoteautorefname{脚注}%
%\newcommand\itemautorefname{项}
\def\figureautorefname{图}
\def\tableautorefname{表}
\def\chapterautorefname{章}
\def\sectionautorefname{节}
\def\subsectionautorefname{小节}
\def\appendixautorefname{附录}
\def\propertyautorefname{性质}
\def\theoremautorefname{定理}
\def\definitionautorefname{定义}
\def\inferenceautorefname{推论}
\def\axiomautorefname{公理}
\def\lemmaautorefname{引理}
\def\principleautorefname{原理}
\def\exerciseautorefname{题目}
\def\topicautorefname{问题}
\def\statementautorefname{命题}

\newcommand\hsp[1][1]{\hspace{#1 em}}
\newcommand\fkh[1][ ]{\nolinebreak\hfill\mbox{(#1}\nolinebreak )}
%选择题括号,用法:\fkh 为右对齐括号,若想往括号内加内容,则用\fkh[XXX]
\newcommand\tk[1][3]{\nolinebreak\mbox{\underline{\hsp[#1]}}}
%填空题横线,用法:\tk 线长默认3em,若想改变长度可用\tk[数字],或直接修改其定义
 
%选项自动排版,用法 \xx{选项}{选项}{选项}{选项},会自动判断排版方式
\newlength\lxxa
\newlength\lxxb
\newlength\lxxc
\newlength\lxxd
\newlength\lxxmax
\newlength\lhalf
\newlength\lhalfhalf
\newcommand\xx[4]{%
\settowidth\lxxa{A. #1}%
\settowidth\lxxb{B. #2}%
\settowidth\lxxc{C. #3}%
\settowidth\lxxd{D. #4}%
\ifthenelse{
    \lengthtest{\lxxa > \lxxb}
}{%
    \setlength\lxxmax\lxxa
}{%
    \setlength\lxxmax\lxxb
}%
\ifthenelse{
    \lengthtest{\lxxmax < \lxxc}
}{%
    \setlength\lxxmax\lxxc
}{}%
\ifthenelse{
    \lengthtest{\lxxmax < \lxxd}
}{%
    \setlength\lxxmax\lxxd
}{}%
\addtolength\lxxmax{1.5em}%
\setlength\lhalf{(\linewidth-\parindent)/2}%
\setlength\lhalfhalf{\lhalf/2}%
\ifthenelse{
    \lengthtest{\lxxmax > \lhalf}
}{%
A. #1 \par B. #2 \par C. #3 \par D. #4%
}{%
    \ifthenelse{
        \lengthtest{\lxxmax > \lhalfhalf}
    }{%
        \begin{minipage}{\lhalf}
        A. #1
        \end{minipage}%
        \begin{minipage}{\lhalf}
        B. #2
        \end{minipage}\par
        \begin{minipage}{\lhalf}
        C. #3
        \end{minipage}%
        \begin{minipage}{\lhalf}
        D. #4
        \end{minipage}%
    }{%
        \begin{minipage}{\lhalfhalf}
        A. #1
        \end{minipage}%
        \begin{minipage}{\lhalfhalf}
        B. #2
        \end{minipage}%
        \begin{minipage}{\lhalfhalf}
        C. #3
        \end{minipage}%
        \begin{minipage}{\lhalfhalf}
        D. #4
        \end{minipage}%
    }%
}}
 
 
%一行四项的图形选项专用,用法 \fourtuxx{图}{图}{图}{图}
\newcommand\fourtuxx[4]{%
\begin{minipage}[b]{(\linewidth-\parindent)/4}
\centering
#1\par A
\end{minipage}%
\begin{minipage}[b]{(\linewidth-\parindent)/4}
\centering
#2\par B
\end{minipage}%
\begin{minipage}[b]{(\linewidth-\parindent)/4}
\centering
#3\par C
\end{minipage}%
\begin{minipage}[b]{(\linewidth-\parindent)/4}
\centering
#4\par D
\end{minipage}}

%%% Local Variables:
%%% mode: latex
%%% TeX-master: "2017"
%%% End:


\title{\kaishu{2017年高考数学试题选析}}
\author{数学解题之路QQ群(60519007)}
\date{更新于: \today}

\begin{document}
\maketitle

\tableofcontents

\section{序言}
\label{sec:preface}

这是2017年高考数学试题的社区解析版,由数学解题之路QQ群(60519007)发起,zhcosin 执笔,这份解析只选取了一部分题目,因为进行全卷解析工作量较大,且我们认为一些基础题目没有必要出现在这份解析中,除非这些题目有些特殊价值。

\section{全国I卷}
\label{sec:nation-1}

\section{全国II卷}
\label{sec:nation-2}

\section{全国III卷}
\label{sec:nation-3}

\section{北京卷}
\label{sec:beijing}

\section{上海卷}
\label{sec:shanghai}

\section{江苏卷}
\label{sec:jiangshu}

\begin{exercise}(第19题)
  对于给定的正整数$k$,若数列$\{a_n\}$满足$a_{n-k}+a_{n-k+1}+\cdots+a_{n-1}+a_{n+1}+\cdots+a_{n+k-1}+a_{n+k}=2ka_n$对任意正整数$n(n>k)$总成立,则称数列$\{a_n\}$是“P(k)数列”,
  \begin{enumerate}
  \item 证明: 等差数列$\{a_n\}$是“$P(k)$数列”.
  \item 若数列$\{a_n\}$既是“$P(2)$数列”,又是“$P(3)$数列”,证明: $\{a_n\}$是等差数列.
  \end{enumerate}
\end{exercise}

\begin{proof}[\textbf{证明}] (北京-陈云恒, 2017-06-08)

  (1). 略.
  (2). 由
  \[ a_{n-3}+a_{n-2}+a_{n-1}+a_{n+1}+a_{n_+2}+a_{n+3}=6a_n \]
  及以下两式
  \begin{eqnarray*}
    a_{n-3}+a_{n-2}+a_n+a_{n+1} & = &4a_{n-1} \\
    a_{n-1}+a_n+a_{n+2}+a_{n+3} & = &4a_{n+1} 
  \end{eqnarray*}
  这两式相加后减去最前面一式即得
  \[ a_{n_1}+a_{n+1}=2a_n \]
  即为等差数列.
\end{proof}

\section{天津卷}
\label{sec:tianjin}

\section{浙江卷}
\label{sec:zhejiang}

\begin{exercise}(第22题)已知数列$\{x_n\}$满足:$x_1=1$,$x_n=x_{n+1}+\ln{(1+x_{n+1})}(n \in N^+)$,证明: 当$n \in N^+$时,
  \begin{enumerate}
  \item $0<x_{n+1}<x_n$.
  \item $2x_{n+1}-x_n \leqslant \frac{x_nx_{n+1}}{2}$.
  \item $\frac{1}{2^{n-1}} \leqslant x_n \leqslant \frac{1}{2^{n-2}}$.
  \end{enumerate}
\end{exercise}

\begin{proof}[\textbf{证明}](北京-陈云恒, 2017-06-08)
  
  (1). 略.

  (2). 由
  
  \[ \ln{(1+x)} \geqslant x-\frac{x^2}{2} \]
  推得
  \[ x_n=x_{n+1}+\ln{(1+x_{n+1})} \geqslant 2x_{n+1}-\frac{x_{n+1}^2}{2} \]
  即
   \[ 2x_{n+1}-x_n \leqslant \frac{x_{n+1}^2}{2} \leqslant \frac{x_nx_{n+1}}{2} \]

   (3). 由第(2)问结果得
   \[ 2 \left( \frac{1}{x_n}-\frac{1}{2} \right) \leqslant \frac{1}{x_{n+1}}- \frac{1}{2} \]
   从而
   \[ \frac{1}{x_n}-\frac{1}{2} \geqslant 2^{n-1} \left( \frac{1}{x_1}-\frac{1}{2} \right) \]
   得
   \[ x_n \leqslant \frac{1}{2^{n-2}} \]
   又
   \[ x_n=x_{n+1}+\ln{(1+x_{n+1})} \leqslant 2x_{n+1} \]
   即
   \[ x_n \geqslant \frac{x_{n-1}}{2} \geqslant \cdots \geqslant \frac{x_1}{2^{n-1}} = \frac{1}{2^{n-1}} \]
   综上,结论得证。
\end{proof}

\begin{proof}[\textbf{证明}] (成都-zhcosin, 2017-06-08)

  (1). 设函数$f(x)=x+\ln{(1+x)}$,求导可知其在$(-1,+\infty)$上递增,且$f(0)=0$,所以如果函数值为正,则自变量也为正,因此由数学归纳法,在$x_n>0$的假设下,便可得出$x_{n+1}>0$,而$x_1=1$,故可知这是一个正项数列,又由$x_{n+1}>0$知$x_n=x_{n+1}+\ln{(1+x_n)}>x_{n+1}$,即数列递减。

  (2). 只需证
  \[ x_n \geqslant \frac{4x_{n+1}}{2+x_{n+1}} \]
  即要证
  \[ x_{n+1}+\ln{(1+x_{n+1})} \geqslant \frac{4x_{n+1}}{2+x_{n+1}} \]
  整理得
  \[ \ln{(1+x_{n+1})} + x_{n+1} + \frac{8}{2+x_{n+1}} -4 \geqslant 0 \]
  因为$x_{n+1}$在区间$(0,1)$内取值,所以这八成就有函数
  \[ h(x)=\ln{(1+x)}+x+\frac{8}{2+x}-4 \]
  在此区间上恒保持正号(实际上区间$(0,1)$的右端点可能还需要限制到更小的$(0,x_2)$上,但这需要对$x_2$进一步估值,高考题通常不至于如此),然而显然$h(0)=0$,并且
  \[ h'(x) = \frac{x(x^2+6x+4)}{(1+x)(2+x)^2} \geqslant 0 \]
  这就得证。

  (第(2)问另证)
  由
  \[ \ln{(1+x)} > x-\frac{x^2}{2} \]
  得
  \[ x_n > 2x_{n+1}-\frac{x_{n+1}^2}{2} \]
  由于$x_{n+1}<1$,所以解得
  \[ x_{n+1} < 2-\sqrt{4-2x_n} = \frac{x_n}{1+\sqrt{1-\frac{x_n}{2}}} \]
  而显然
  \[ \sqrt{1-\frac{x_n}{2}} > 1-\frac{x_n}{2} \]
  所以
  \[ x_{n+1} < \frac{x_n}{2-\frac{x_n}{2}} \]
  即
  \[ 2x_{n+1}-x_n \leqslant \frac{x_nx_{n+1}}{2} \]
   (3). 由熟知的不等式$\ln{(1+x)}<x$对任意$x>-1$成立,所以得到$x_n<2x_{n+1}$,于是不等式左边便得证,至于右边,利用第二问结论可得
  \[ \frac{1}{2^nx_n}-\frac{1}{2^{n+1}x_{n+1}} \leqslant \frac{1}{2^{n+2}} \]
  于是
  \[ \frac{1}{2x_1} - \frac{1}{2^nx_n} = \sum_{k=1}^{n-1} \left( \frac{1}{2^kx_k}-\frac{1}{2^{k+1}x_{k+1}} \right) \leqslant \sum_{k=1}^{n-1}\frac{1}{2^{k+2}}=\frac{1}{4}-\frac{1}{2^{n+1}} \]
  由此得
  \[ x_n \leqslant \frac{1}{2^{n-2}+\frac{1}{2}} <\frac{1}{2^{n-2}} \]
\end{proof}

\section{山东卷}
\label{sec:shandong}

\section{试题研究}
\label{sec:problem-study}

\subsection{江苏卷数列题目结论的推广}
\label{sec:jiangshu-series-extend}

\textbf{作者}: 成都-zhcosin

江苏卷的$P(k)$数列题目的结论可以推广为,设$r$和$s$是两个不同的正整数,且都大于1,若某数列既是$P(r)$数列又是$P(s)$数列,则它一定是等差数列。

下面的证明过程中有一个引理暂未能得到定理,如果有谁证明了它,请告诉我一下(QQ:532319166).

\begin{proof}[\textbf{证明}]
  设数列是$P(m)$数列,则有递推公式
  \[ a_{n-m}+a_{n-m+1}+\cdots+a_{n-1}-2ma_n+a_{n+1}+\cdots+a_{n-m+1}+a_{n+m}=0 \]
  这是齐次线性递推数列,其特征方程是
  \[ 1+x+\cdots+x^{m-1}-2mx^m+x^{m+1}+\cdots+x^{2m}=0 \]
  为了得到它的通项表示,讨论它的根的情况,记
    \[ f(x)= 1+x+\cdots+x^{m-1}-2mx^m+x^{m+1}+\cdots+x^{2m}\]
  因为$f(1)=f'(1)=0$,所以$x=1$是它的二重根。设它的其它复数根分别为$t_i(i=2,2,\cdots,q)$,则通项可表为
  \[ a_n=c_0+c_1 n + \sum_{i=2}^q p_i(n)t_i^n \]
  其中的系数$c_0,c_1$为复常数,$p_i(n)$是一些关于$n$的多项式。

  如果数列同时是$P(r)$数列和$P(s)$数列,那么数列的通项将有两种上述表达式,要证明它是等差数列,我们只要证明如下的另一个引理:
  \begin{lemma}
    设$t_1,t_2,\cdots,t_m$是$m$个非零且互不相等的复常数,而$p_i(x)(i=1,2,\cdots,m)$一些关于$x$的多项式,如果等式$p_1(n)t_1^n+p_2(n)t_2^n+\cdots+p_m(n)t_m^n=0$对一切正整数$n$都成立,则$p_i(x)(i=1,2,\ldots,n)$全是零多项式。
  \end{lemma}
  证明如下:

  由条件知有如下等式
  \begin{equation*}
    \begin{pmatrix}
      t_1 & t_2 & \cdots & t_m \\
      t_1^2 & t_2^2 & \cdots & t_{m^2} \\
      \cdots \\
      t_1^r & t_2^r & \cdots & t_m^r
    \end{pmatrix}
    \cdot
    \begin{pmatrix}
      p_1(1) & p_1(2) & \cdots & p_1(r) \\
      p_2(1) & p_2(2) & \cdots & p_2(r) \\
      \cdots \\
      p_m(1) & p_m(2) & \cdots & p_m(r)
    \end{pmatrix}
    = O
  \end{equation*}
  上式中$r$可以一直写到任意正整数,由上式便知,$p_i(k)(i=1,2,\ldots,m)$是下面齐次线性方程的解
  \begin{equation*}
    \left\{
        \begin{array}{l}
         p_1(k)t_1+p_2(k)t_2+\cdots+p_m(k)t_m=0 \\
         p_1(k)t_1^2+p_2(k)t_2^2+\cdots+p_m(k)t_m^2=0 \\
          \cdots \\
         p_1(k)t_1^r+p_2(k)t_2^r+\cdots+p_m(k)t_m^r=0 \\
        \end{array}
        \right.
  \end{equation*}
  其中$k=1,2,\ldots$,取$r=m$时,方程个数与未知数个数相同,由范德蒙行列式可得
  \begin{equation*}
    \begin{vmatrix}
      t_1 & t_2 & \cdots & t_m \\
      t_1^2 & t_2^2 & \cdots & t_{m^2} \\
      \cdots \\
      t_1^m & t_2^m & \cdots & t_m^m
    \end{vmatrix}
    = t_1t_2\cdots t_m \prod_{1 \leqslant i < j \leqslant m}(t_i-t_j) \neq 0
  \end{equation*}
  因此方程的系数行列式非零,故上述方程只有零解,即
  \[ p_1(k)=0, p_2(k)=0, \cdots, p_m(k)=0 \]
  而$k$又是任意的正整数,所以$p_i(x)(i=1,2,\ldots,m)$只能全是零多项式。于是得证.
\end{proof}

\end{document}

%%% Local Variables:
%%% mode: latex
%%% TeX-master: t
%%% End:
