
\documentclass{ctexart}
%\usepackage{minitoc}

% 定制页面版式的宏包
\usepackage[a4paper,left=3.5cm,right=3.5cm, bottom=3.5cm,top=3.5cm]{geometry}

% 定制目录样式的宏包
\usepackage{etoc}

% 定制日期时间格式的宏包
\usepackage[yyyymmdd]{datetime}
\renewcommand{\dateseparator}{-}

% 数学必备宏包
\usepackage{amsmath}
\usepackage{amssymb}

% 定理和证明环境
\usepackage{amsthm}

\usepackage{makecell}

% 插图宏包,提供插图命令 \includegraphics
\usepackage{graphicx}

% 好看的向量箭头符号,命令是 \vv
\usepackage{esvect}

% 数学花体,命令是\mathscr
\usepackage{mathrsfs}

% 数学粗体,用于向量或矩阵等,命令是\bm
\usepackage{bm}

% 改善表格排版质量的宏包
\usepackage{booktabs}

% 使目录和各种引用具有超链接效果
\usepackage[colorlinks,linkcolor=black,CJKbookmarks=true,bookmarksnumbered]{hyperref}

% 定制插图和表格的标题的宏包
\usepackage[font=small,labelfont=bf,labelsep=space]{caption}


%%% Local Variables:
%%% mode: latex
%%% TeX-master: "book"
%%% End:


% 自定义环境
\newcounter{example}[section]
\renewcommand{\theexample}{\thesection.\arabic{example}}

\newenvironment{example}[1][]{\refstepcounter{example} \textbf{例 \theexample  \ #1} \hspace{0.5em}}{\hspace{\stretch{1}} \rule{1ex}{1ex}}

\newcounter{exercise}[section]
\renewcommand{\theexercise}{\thesection.\arabic{exercise}}

\newenvironment{exercise}[1][]{\refstepcounter{exercise} \textbf{试题 \theexercise  \ #1} \hspace{0.5em}}{}

\newtheorem{definition}{定义}[section]
\newtheorem{property}{性质}[section]
\newtheorem{theorem}{定理}[section]
\newtheorem{inference}{推论}[section]
\newtheorem{axiom}{公理}[section]
\newtheorem{lemma}{引理}[section]
\newtheorem{principle}{原理}[section]
%\newtheorem{exercise}{题目}[section]
\newtheorem{topic}{问题}[section]
\newtheorem{statement}{命题}[section]
% \newtheorem{example}{例}[section]

% 使公式编号与章节关联,命令由 amsmath 宏包提供
\numberwithin{equation}{section}

% 配合 \autoref 命令使引用不只引用编号,也能引用环境命名,如: 定理 3.2.5,来自 hyperref 宏包。
%\newcommand\equationautorefname{式}
%\newcommand\footnoteautorefname{脚注}%
%\newcommand\itemautorefname{项}
\def\figureautorefname{图}
\def\tableautorefname{表}
\def\chapterautorefname{章}
\def\sectionautorefname{节}
\def\subsectionautorefname{小节}
\def\appendixautorefname{附录}
\def\propertyautorefname{性质}
\def\theoremautorefname{定理}
\def\definitionautorefname{定义}
\def\inferenceautorefname{推论}
\def\axiomautorefname{公理}
\def\lemmaautorefname{引理}
\def\principleautorefname{原理}
\def\exerciseautorefname{题目}
\def\topicautorefname{问题}
\def\statementautorefname{命题}

%%% Local Variables:
%%% mode: latex
%%% TeX-master: "2017"
%%% End:


\title{\kaishu{2017年高考数学试题选析}}
\author{数学解题之路QQ群(60519007)}
\date{更新于: \today}

\begin{document}
\maketitle

\tableofcontents

\section{序言}
\label{sec:preface}

这是2017年高考数学试题的社区解析版,由数学解题之路QQ群(60519007)发起,zhcosin 执笔,这份解析只选取了一部分题目,因为进行全卷解析工作量较大,且我们认为一些基础题目没有必要出现在这份解析中,除非这些题目有些特殊价值。

\section{全国I卷}
\label{sec:nation-1}

\section{全国II卷}
\label{sec:nation-2}

\section{全国III卷}
\label{sec:nation-3}

\section{北京卷}
\label{sec:beijing}

\section{上海卷}
\label{sec:shanghai}

\section{江苏卷}
\label{sec:jiangshu}

\begin{exercise}(第19题)
  对于给定的正整数$k$,若数列$\{a_n\}$满足$a_{n-k}+a_{n-k+1}+\cdots+a_{n-1}+a_{n+1}+\cdots+a_{n+k-1}+a_{n+k}=2ka_n$对任意正整数$n(n>k)$总成立,则称数列$\{a_n\}$是“P(k)数列”,
  \begin{enumerate}
  \item 证明: 等差数列$\{a_n\}$是“$P(k)$数列”.
  \item 若数列$\{a_n\}$既是“$P(2)$数列”,又是“$P(3)$数列”,证明: $\{a_n\}$是等差数列.
  \end{enumerate}
\end{exercise}

\begin{proof}[\textbf{证明}] (北京-陈云恒, 2017-06-08)

  (1). 略.
  (2). 由
  \[ a_{n-3}+a_{n-2}+a_{n-1}+a_{n+1}+a_{n_+2}+a_{n+3}=6a_n \]
  及以下两式
  \begin{eqnarray*}
    a_{n-3}+a_{n-2}+a_n+a_{n+1} & = &4a_{n-1} \\
    a_{n-1}+a_n+a_{n+2}+a_{n+3} & = &4a_{n+1} 
  \end{eqnarray*}
  这两式相加后减去最前面一式即得
  \[ a_{n_1}+a_{n+1}=2a_n \]
  即为等差数列.
\end{proof}

\section{天津卷}
\label{sec:tianjin}

\section{浙江卷}
\label{sec:zhejiang}

\begin{exercise}(第22题)已知数列$\{x_n\}$满足:$x_1=1$,$x_n=x_{n+1}+\ln{(1+x_{n+1})}(n \in N^+)$,证明: 当$n \in N^+$时,
  \begin{enumerate}
  \item $0<x_{n+1}<x_n$.
  \item $2x_{n+1}-x_n \leqslant \frac{x_nx_{n+1}}{2}$.
  \item $\frac{1}{2^{n-1}} \leqslant x_n \leqslant \frac{1}{2^{n-2}}$.
  \end{enumerate}
\end{exercise}

\begin{proof}[\textbf{证明}](北京-陈云恒, 2017-06-08)
  
  (1). 略.

  (2). 由
  
  \[ \ln{(1+x)} \geqslant x-\frac{x^2}{2} \]
  推得
  \[ x_n=x_{n+1}+\ln{(1+x_{n+1})} \geqslant 2x_{n+1}-\frac{x_{n+1}^2}{2} \]
  即
   \[ 2x_{n+1}-x_n \leqslant \frac{x_{n+1}^2}{2} \leqslant \frac{x_nx_{n+1}}{2} \]

   (3). 由第(2)问结果得
   \[ 2 \left( \frac{1}{x_n}-\frac{1}{2} \right) \leqslant \frac{1}{x_{n+1}}- \frac{1}{2} \]
   从而
   \[ \frac{1}{x_n}-\frac{1}{2} \geqslant 2^{n-1} \left( \frac{1}{x_1}-\frac{1}{2} \right) \]
   得
   \[ x_n \leqslant \frac{1}{2^{n-2}} \]
   又
   \[ x_n=x_{n+1}+\ln{(1+x_{n+1})} \leqslant 2x_{n+1} \]
   即
   \[ x_n \geqslant \frac{x_{n-1}}{2} \geqslant \cdots \geqslant \frac{x_1}{2^{n-1}} = \frac{1}{2^{n-1}} \]
   综上,结论得证。
\end{proof}

\begin{proof}[\textbf{证明}] (成都-zhcosin, 2017-06-08)

  (1). 设函数$f(x)=x+\ln{(1+x)}$,求导可知其在$(-1,+\infty)$上递增,且$f(0)=0$,所以如果函数值为正,则自变量也为正,因此由数学归纳法,在$x_n>0$的假设下,便可得出$x_{n+1}>0$,而$x_1=1$,故可知这是一个正项数列,又由$x_{n+1}>0$知$x_n=x_{n+1}+\ln{(1+x_n)}>x_{n+1}$,即数列递减。

  (2). 只需证
  \[ x_n \geqslant \frac{4x_{n+1}}{2+x_{n+1}} \]
  即要证
  \[ x_{n+1}+\ln{(1+x_{n+1})} \geqslant \frac{4x_{n+1}}{2+x_{n+1}} \]
  整理得
  \[ \ln{(1+x_{n+1})} + x_{n+1} + \frac{8}{2+x_{n+1}} -4 \geqslant 0 \]
  因为$x_{n+1}$在区间$(0,1)$内取值,所以这八成就有函数
  \[ h(x)=\ln{(1+x)}+x+\frac{8}{2+x}-4 \]
  在此区间上恒保持正号(实际上区间$(0,1)$的右端点可能还需要限制到更小的$(0,x_2)$上,但这需要对$x_2$进一步估值,高考题通常不至于如此),然而显然$h(0)=0$,并且
  \[ h'(x) = \frac{x(x^2+6x+4)}{(1+x)(2+x)^2} \geqslant 0 \]
  这就得证。

  (第(2)问另证)
  由
  \[ \ln{(1+x)} > x-\frac{x^2}{2} \]
  得
  \[ x_n > 2x_{n+1}-\frac{x_{n+1}^2}{2} \]
  由于$x_{n+1}<1$,所以解得
  \[ x_{n+1} < 2-\sqrt{4-2x_n} = \frac{x_n}{1+\sqrt{1-\frac{x_n}{2}}} \]
  而显然
  \[ \sqrt{1-\frac{x_n}{2}} > 1-\frac{x_n}{2} \]
  所以
  \[ x_{n+1} < \frac{x_n}{2-\frac{x_n}{2}} \]
  即
  \[ 2x_{n+1}-x_n \leqslant \frac{x_nx_{n+1}}{2} \]
   (3). 由熟知的不等式$\ln{(1+x)}<x$对任意$x>-1$成立,所以得到$x_n<2x_{n+1}$,于是不等式左边便得证,至于右边,利用第二问结论可得
  \[ \frac{1}{2^nx_n}-\frac{1}{2^{n+1}x_{n+1}} \leqslant \frac{1}{2^{n+2}} \]
  于是
  \[ \frac{1}{2x_1} - \frac{1}{2^nx_n} = \sum_{k=1}^{n-1} \left( \frac{1}{2^kx_k}-\frac{1}{2^{k+1}x_{k+1}} \right) \leqslant \sum_{k=1}^{n-1}\frac{1}{2^{k+2}}=\frac{1}{4}-\frac{1}{2^{n+1}} \]
  由此得
  \[ x_n \leqslant \frac{1}{2^{n-2}+\frac{1}{2}} <\frac{1}{2^{n-2}} \]
\end{proof}

\section{山东卷}
\label{sec:shandong}

\end{document}

%%% Local Variables:
%%% mode: latex
%%% TeX-master: t
%%% End:
