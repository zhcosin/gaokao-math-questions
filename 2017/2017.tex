
\documentclass{ctexart}
%\usepackage{minitoc}

% 定制页面版式的宏包
\usepackage[a4paper,left=3.5cm,right=3.5cm, bottom=3.5cm,top=3.5cm]{geometry}

% 定制目录样式的宏包
\usepackage{etoc}

% 定制日期时间格式的宏包
\usepackage[yyyymmdd]{datetime}
\renewcommand{\dateseparator}{-}

% 数学必备宏包
\usepackage{amsmath}
\usepackage{amssymb}

% 定理和证明环境
\usepackage{amsthm}

\usepackage{makecell}

% 插图宏包,提供插图命令 \includegraphics
\usepackage{graphicx}

% 好看的向量箭头符号,命令是 \vv
\usepackage{esvect}

% 数学花体,命令是\mathscr
\usepackage{mathrsfs}

% 数学粗体,用于向量或矩阵等,命令是\bm
\usepackage{bm}

% 改善表格排版质量的宏包
\usepackage{booktabs}

% 使目录和各种引用具有超链接效果
\usepackage[colorlinks,linkcolor=black,CJKbookmarks=true,bookmarksnumbered]{hyperref}

% 定制插图和表格的标题的宏包
\usepackage[font=small,labelfont=bf,labelsep=space]{caption}


%%% Local Variables:
%%% mode: latex
%%% TeX-master: "book"
%%% End:


% 自定义环境
\newcounter{example}[section]
\renewcommand{\theexample}{\thesection.\arabic{example}}

\newenvironment{example}[1][]{\refstepcounter{example} \textbf{例 \theexample  \ #1} \hspace{0.5em}}{\hspace{\stretch{1}} \rule{1ex}{1ex}}

\newcounter{exercise}[section]
\renewcommand{\theexercise}{\thesection.\arabic{exercise}}

\newenvironment{exercise}[1][]{\refstepcounter{exercise} \textbf{试题 \theexercise  \ #1} \hspace{0.5em}}{}

\newtheorem{definition}{定义}[section]
\newtheorem{property}{性质}[section]
\newtheorem{theorem}{定理}[section]
\newtheorem{inference}{推论}[section]
\newtheorem{axiom}{公理}[section]
\newtheorem{lemma}{引理}[section]
\newtheorem{principle}{原理}[section]
%\newtheorem{exercise}{题目}[section]
\newtheorem{topic}{问题}[section]
\newtheorem{statement}{命题}[section]
% \newtheorem{example}{例}[section]

% 使公式编号与章节关联,命令由 amsmath 宏包提供
\numberwithin{equation}{section}

% 配合 \autoref 命令使引用不只引用编号,也能引用环境命名,如: 定理 3.2.5,来自 hyperref 宏包。
%\newcommand\equationautorefname{式}
%\newcommand\footnoteautorefname{脚注}%
%\newcommand\itemautorefname{项}
\def\figureautorefname{图}
\def\tableautorefname{表}
\def\chapterautorefname{章}
\def\sectionautorefname{节}
\def\subsectionautorefname{小节}
\def\appendixautorefname{附录}
\def\propertyautorefname{性质}
\def\theoremautorefname{定理}
\def\definitionautorefname{定义}
\def\inferenceautorefname{推论}
\def\axiomautorefname{公理}
\def\lemmaautorefname{引理}
\def\principleautorefname{原理}
\def\exerciseautorefname{题目}
\def\topicautorefname{问题}
\def\statementautorefname{命题}

%%% Local Variables:
%%% mode: latex
%%% TeX-master: "2017"
%%% End:


\title{\kaishu{2017年高考数学试题选析}}
\author{数学解题之路QQ群(60519007)}
\date{更新于: \today}

\begin{document}
\maketitle

\tableofcontents

\section{序言}
\label{sec:preface}

这是2017年高考数学试题的社区解析版,由数学解题之路QQ群(60519007)发起,zhcosin 执笔,这份解析只选取了一部分题目,因为进行全卷解析工作量较大,且我们认为一些基础题目没有必要出现在这份解析中,除非这些题目有些特殊价值。

\section{全国I卷}
\label{sec:nation-1}

\begin{exercise}(理科第12题)
  几位大学生响应国家的创业号召,开发了一款应用软件,为激发大家学习数学的兴趣,他们推出了“解数学题获取软件激活码”的活动。这款软件的激活码为下面数学问题的答案:已知数列$1,1,2,1,2,4,1,2,4,8,1,2,4,8,16,\ldots$,其中第一项是$2^0$,接下来的两项是$2^0,2^1$,再接下来的三项是$2^0,2^1,2^2$,依此类推。求满足如下条件的最小正整数$N$: $N>100$且该数列的前$N$项和为2的整数幂。那么该软件的激活码是 \fkh 

  \xx{440}{330}{220}{110}
\end{exercise}

\begin{exercise}(理科第20题)
  已知椭圆$C$: $\frac{x^2}{a^2}+\frac{y^2}{b^2}=1(a>b>0)$,四点$P_1(1,1)$、$P_2(0,1)$、$P_3(-1,\frac{\sqrt{3}}{2})$、$P_4(1,\frac{\sqrt{3}}{2})$中恰有三点在椭圆上.
  \begin{enumerate}
  \item 求$C$的方程.
  \item 设直线$l$不经过$P_2$点且与$C$相交于$A$、$B$两点. 若直线$P_2A$与直线$P_2B$的斜率之和为-1,证明: $l$过定点.
  \end{enumerate}
\end{exercise}

\begin{exercise}(理科第21题)
  已知函数$f(x)=ae^{2x}+(a-2)e^x-x$.
  \begin{enumerate}
  \item 讨论$f(x)$的单调性.
  \item 若$f(x)$有两个零点,求$a$的取值范围.
  \end{enumerate}
\end{exercise}

\section{全国II卷}
\label{sec:nation-2}

\begin{exercise}(理科第12题)
  已知$\triangle ABC$是边长为2的正三角形,$P$为平面$ABC$内一点,则$\vv{PA} \cdot (\vv{PB}+\vv{PC})$的最小值是(    )

  \xx{$-2$}{$-\frac{3}{2}$}{$-\frac{4}{3}$}{$-1$}
\end{exercise}

\begin{figure}[htbp]
  \centering
\includegraphics{pic/nation-2-like-12.pdf}
\caption{}
\label{fig:nation-2-like-12}
\end{figure}

\begin{proof}[\textbf{解答}](广州-kuing, 2017-06-10)

  取$BC$边中点$D$,则
  \begin{eqnarray*}
    \vv{PA} \cdot (\vv{PB}+\vv{PC}) & = &  \vv{PA} \cdot 2 \vv{PD} \\
                                    & = & \frac{1}{2} \big( (\vv{PA}+\vv{PD})^2 + (\vv{PA}-\vv{PD})^2 \big) \\
    & \geqslant & -\frac{1}{2}(\vv{PA}-\vv{PD})^2 = -\frac{1}{2}\vv{AD}^2 = - \frac{3}{2}
  \end{eqnarray*}
\end{proof}

\begin{proof}[\textbf{解答}](成都-zhcosin, 2017-06-09)

  如\autoref{fig:nation-2-like-12},取$BC$边中点为$D$,则$\vv{PB}+\vv{PC}=2\vv{PD}$,于是原式即为$2\vv{PA}\cdot \vv{PD}=PA^2+PD^2-AD^2 \geqslant \frac{1}{2}(PA+PD)^2 - AD^2 \geqslant \frac{1}{2}AD^2 -  AD^2 = - \frac{1}{2} AD^2 = -\frac{3}{2}$.
\end{proof}

\begin{proof}[\textbf{解答}](成都-zhcosin, 2017-06-09)

  把所有向量都用$\vv{AB}$和$\vv{AC}$来表示,设$\vv{PA}=x\vv{AB}+y\vv{AC}$,则$\vv{PB}=\vv{PA}+\vv{AB}=(1+x)\vv{AB}+y\vv{AC}$,$\vv{PC}=\vv{PA}+\vv{AC}=x\vv{AB}+(1+y)\vv{AC}$,于是
  \begin{eqnarray*}
    \vv{PA} \cdot (\vv{PB}+\vv{PC}) & = & (x\vv{AB}+y\vv{AC}) \cdot ((1+2x)\vv{AB}+(1+2y)\vv{AC}) \\
                                    & = & x(1+2x)\vv{AB}^2+y(1+2y)\vv{AC}^2+(x+y+4xy)(\vv{AB} \cdot \vv{AC}) \\
                                    & = & 4x(1+2x)+4y(1+2y)+2(x+y+4xy) \\
    & = & 8x^2+8y^2+8xy+6x+6y
  \end{eqnarray*}
  为求此式最小值,作代换
  \[ x=u+v, \  y=u-v \]
  于是上式成为
  \begin{eqnarray*}
    & & 8(u+v)^2+8(u-v)^2+8(u+v)(u-v)+12u \\
    & = & 24u^2+12u+8v^2 \\
    & \geqslant & 24u^2+12u \\
    & = & 24 \left( u+\frac{1}{4} \right)^2-\frac{3}{2} \geqslant -\frac{3}{2}
  \end{eqnarray*}
\end{proof}

\begin{exercise}(理科第21题)
  已知函数$f(x)=ax^2-ax-x\ln{x}$,且$f(x) \geqslant 0$.
  \begin{enumerate}
  \item 求$a$;
  \item 证明: $f(x)$存在唯一的极大值点$x_0$,且$e^{-2}<f(x_0)<2^{-2}$.
  \end{enumerate}
\end{exercise}

\section{全国III卷}
\label{sec:nation-3}

\begin{exercise}(理科第21题)
  已知函数$f(x)=x-1-a\ln{x}$,
  \begin{enumerate}
  \item 若$f(x) \geqslant 0$,求$a$的值.
  \item 设$m$为整数,且对于任意正整数$n$,$\left( 1+\frac{1}{2} \right)\left( 1+\frac{1}{2^2} \right) \cdots \left( 1+\frac{1}{2^n} \right) <m$,求$m$的最小值。
  \end{enumerate}
\end{exercise}

\begin{proof}[\textbf{解答}](山西-张沛, 2017-06-08)

  (1). 略,答案$a=1$.

  (2). 对左边取对数,并利用当$x>-1$时$\ln{(1+x)}<x$,有
  \[ \sum_{i=1}^n \ln{\left( 1+\frac{1}{2^i} \right)} < \sum_{i=1}^n \frac{1}{2^i} = 1-\frac{1}{2^n} < 1 \]
  所以原式小于$e$,又
  \[ \left( 1+\frac{1}{2} \right)\left( 1+\frac{1}{2^2} \right)\left( 1+\frac{1}{2^3} \right) = \frac{135}{64}>2 \]
  因此$m$的最小值是3.
\end{proof}

\section{北京卷}
\label{sec:beijing}

\section{上海卷}
\label{sec:shanghai}

\section{江苏卷}
\label{sec:jiangshu}

\begin{exercise}(第19题)
  对于给定的正整数$k$,若数列$\{a_n\}$满足$a_{n-k}+a_{n-k+1}+\cdots+a_{n-1}+a_{n+1}+\cdots+a_{n+k-1}+a_{n+k}=2ka_n$对任意正整数$n(n>k)$总成立,则称数列$\{a_n\}$是“P(k)数列”,
  \begin{enumerate}
  \item 证明: 等差数列$\{a_n\}$是“$P(k)$数列”.
  \item 若数列$\{a_n\}$既是“$P(2)$数列”,又是“$P(3)$数列”,证明: $\{a_n\}$是等差数列.
  \end{enumerate}
\end{exercise}

\begin{proof}[\textbf{证明}] (北京-陈云恒, 2017-06-08)

  (1). 略.
  (2). 由
  \[ a_{n-3}+a_{n-2}+a_{n-1}+a_{n+1}+a_{n_+2}+a_{n+3}=6a_n \]
  及以下两式
  \begin{eqnarray*}
    a_{n-3}+a_{n-2}+a_n+a_{n+1} & = &4a_{n-1} \\
    a_{n-1}+a_n+a_{n+2}+a_{n+3} & = &4a_{n+1} 
  \end{eqnarray*}
  这两式相加后减去最前面一式即得
  \[ a_{n_1}+a_{n+1}=2a_n \]
  即为等差数列.
\end{proof}

\section{天津卷}
\label{sec:tianjin}

\section{浙江卷}
\label{sec:zhejiang}

\begin{exercise}(第22题)已知数列$\{x_n\}$满足:$x_1=1$,$x_n=x_{n+1}+\ln{(1+x_{n+1})}(n \in N^+)$,证明: 当$n \in N^+$时,
  \begin{enumerate}
  \item $0<x_{n+1}<x_n$.
  \item $2x_{n+1}-x_n \leqslant \frac{x_nx_{n+1}}{2}$.
  \item $\frac{1}{2^{n-1}} \leqslant x_n \leqslant \frac{1}{2^{n-2}}$.
  \end{enumerate}
\end{exercise}

\begin{proof}[\textbf{证明}](北京-陈云恒, 2017-06-08)
  
  (1). 略.

  (2). 由
  
  \[ \ln{(1+x)} \geqslant x-\frac{x^2}{2} \]
  推得
  \[ x_n=x_{n+1}+\ln{(1+x_{n+1})} \geqslant 2x_{n+1}-\frac{x_{n+1}^2}{2} \]
  即
   \[ 2x_{n+1}-x_n \leqslant \frac{x_{n+1}^2}{2} \leqslant \frac{x_nx_{n+1}}{2} \]

   (3). 由第(2)问结果得
   \[ 2 \left( \frac{1}{x_n}-\frac{1}{2} \right) \leqslant \frac{1}{x_{n+1}}- \frac{1}{2} \]
   从而
   \[ \frac{1}{x_n}-\frac{1}{2} \geqslant 2^{n-1} \left( \frac{1}{x_1}-\frac{1}{2} \right) \]
   得
   \[ x_n \leqslant \frac{1}{2^{n-2}} \]
   又
   \[ x_n=x_{n+1}+\ln{(1+x_{n+1})} \leqslant 2x_{n+1} \]
   即
   \[ x_n \geqslant \frac{x_{n-1}}{2} \geqslant \cdots \geqslant \frac{x_1}{2^{n-1}} = \frac{1}{2^{n-1}} \]
   综上,结论得证。
\end{proof}

\begin{proof}[\textbf{证明}] (成都-zhcosin, 2017-06-08)

  (1). 设函数$f(x)=x+\ln{(1+x)}$,求导可知其在$(-1,+\infty)$上递增,且$f(0)=0$,所以如果函数值为正,则自变量也为正,因此由数学归纳法,在$x_n>0$的假设下,便可得出$x_{n+1}>0$,而$x_1=1$,故可知这是一个正项数列,又由$x_{n+1}>0$知$x_n=x_{n+1}+\ln{(1+x_n)}>x_{n+1}$,即数列递减。

  (2). 只需证
  \[ x_n \geqslant \frac{4x_{n+1}}{2+x_{n+1}} \]
  即要证
  \[ x_{n+1}+\ln{(1+x_{n+1})} \geqslant \frac{4x_{n+1}}{2+x_{n+1}} \]
  整理得
  \[ \ln{(1+x_{n+1})} + x_{n+1} + \frac{8}{2+x_{n+1}} -4 \geqslant 0 \]
  因为$x_{n+1}$在区间$(0,1)$内取值,所以这八成就有函数
  \[ h(x)=\ln{(1+x)}+x+\frac{8}{2+x}-4 \]
  在此区间上恒保持正号(实际上区间$(0,1)$的右端点可能还需要限制到更小的$(0,x_2)$上,但这需要对$x_2$进一步估值,高考题通常不至于如此),然而显然$h(0)=0$,并且
  \[ h'(x) = \frac{x(x^2+6x+4)}{(1+x)(2+x)^2} \geqslant 0 \]
  这就得证。

  (第(2)问另证)
  由
  \[ \ln{(1+x)} > x-\frac{x^2}{2} \]
  得
  \[ x_n > 2x_{n+1}-\frac{x_{n+1}^2}{2} \]
  由于$x_{n+1}<1$,所以解得
  \[ x_{n+1} < 2-\sqrt{4-2x_n} = \frac{x_n}{1+\sqrt{1-\frac{x_n}{2}}} \]
  而显然
  \[ \sqrt{1-\frac{x_n}{2}} > 1-\frac{x_n}{2} \]
  所以
  \[ x_{n+1} < \frac{x_n}{2-\frac{x_n}{2}} \]
  即
  \[ 2x_{n+1}-x_n \leqslant \frac{x_nx_{n+1}}{2} \]
   (3). 由熟知的不等式$\ln{(1+x)}<x$对任意$x>-1$成立,所以得到$x_n<2x_{n+1}$,于是不等式左边便得证,至于右边,利用第二问结论可得
  \[ \frac{1}{2^nx_n}-\frac{1}{2^{n+1}x_{n+1}} \leqslant \frac{1}{2^{n+2}} \]
  于是
  \[ \frac{1}{2x_1} - \frac{1}{2^nx_n} = \sum_{k=1}^{n-1} \left( \frac{1}{2^kx_k}-\frac{1}{2^{k+1}x_{k+1}} \right) \leqslant \sum_{k=1}^{n-1}\frac{1}{2^{k+2}}=\frac{1}{4}-\frac{1}{2^{n+1}} \]
  由此得
  \[ x_n \leqslant \frac{1}{2^{n-2}+\frac{1}{2}} <\frac{1}{2^{n-2}} \]
\end{proof}

\section{山东卷}
\label{sec:shandong}

\newpage

\section{关于全国III卷压轴题的讨论}
\label{sec:discussion-about-nation-3-infty-product}

\textbf{作者}: 成都-zhcosin, 2017-06-10

全国III卷的压轴题并不难,我所感兴趣的是那个无穷乘积
\[ \prod_{n=1}^{\infty}\left( 1+\frac{1}{2^n} \right) \]
本来试图求出这个无穷乘积来,但是没有成功,只表成了另一个无穷级数的和。

对这无穷乘积取对数,便得级数
\[ \sum_{n=1}^{\infty} \ln{ \left( 1+\frac{1}{2^n} \right) } \]
由$\ln(1+x)<x$及级数$\sum_{n=1}^{\infty} \frac{1}{2^n}$收敛,知上面这个级数也收敛,从而前面的无穷乘积也收敛。

首先根据函数$\ln{(1+x)}$的泰勒级数
\[ \ln{(1+x)} = x-\frac{x^2}{2}+\frac{x^3}{3}+\cdots + (-1)^{n-1} \frac{x^n}{n}  + \cdots \]
记
\[ T_n(x) = \sum_{k=1}^n (-1)^{k-1}\frac{x^k}{k} \]
则有如下结论:
\begin{statement}
  设$x>0$,当$n$为奇数时, $\ln{(1+x)} < T_n(x)$,而$n$为偶数时, $\ln{(1+x)}>T_n(x)$.
\end{statement}
实际上由拉格朗日余项的符号便可立得此结论,但此利用导数证明一下(也就是不超出高中知识范围).
\begin{proof}[\textbf{证明}]
  对$T_n(x)$求导得
  \[ T_n'(x) = 1-x+x^2-\cdots + (-x)^{n-1} = \frac{1-(-x)^n}{1+x} \]
  记
  \[ R_n(x) = \ln{(1+x)}-T_n(x) \]
  则
  \[ R_n'(x) = \frac{1}{1+x} - \frac{1-(-x)^n}{1+x} = \frac{(-x)^n}{1+x} \]
  而$R_n(0)=0$,所以当$n$为奇数时,$R_n(x)$在正实数区间上单调增加,而$n$为偶数时则为单调递减,故得结论。
\end{proof}

于是得如下逼近式(式中$r$和$s$是任意两个正整数)
\[ T_{2r}(x) < \ln{(1+x)} < T_{2s+1}(x) \]
取$x=\frac{1}{2^u}$,得
\[ \sum_{k=1}^{2r} (-1)^{k-1} \frac{1}{k(2^u)^k} < \ln{\left( 1+\frac{1}{2^u} \right) < \sum_{k=1}^{2s+1} (-1)^{k-1} \frac{1}{k(2^u)^k}} \]
再对$u=1,2,\ldots,n$进行累加得
\[ \sum_{u=1}^n \sum_{k=1}^{2r} (-1)^{k-1} \frac{1}{k(2^u)^k} < \sum_{u=1}^n \ln{\left( 1+\frac{1}{2^u} \right) < \sum_{u=1}^{n} \sum_{k=1}^{2s+1} (-1)^{k-1} \frac{1}{k(2^u)^k}} \]
两边的二重求和交换求和顺序,得
\[  \sum_{k=1}^{2r} \sum_{u=1}^n (-1)^{k-1} \frac{1}{k(2^k)^u} < \sum_{u=1}^n \ln{\left( 1+\frac{1}{2^u} \right) <  \sum_{k=1}^{2s+1} \sum_{u=1}^{n}(-1)^{k-1} \frac{1}{k(2^k)^u}} \]
即
\[ \sum_{k=1}^{2r} (-1)^{k-1} \frac{1}{k(2^k-1)} \left( 1-\frac{1}{2^{nk}} \right) < \sum_{u=1}^n \ln{\left( 1+\frac{1}{2^u} \right)} < \sum_{k=1}^{2s+1} (-1)^{k-1} \frac{1}{k(2^k-1)} \left( 1-\frac{1}{2^{nk}} \right)\]
令$n\to\infty$取极限,得
\[ \sum_{k=1}^{2r} (-1)^{k-1} \frac{1}{k(2^k-1)} \leqslant \sum_{u=1}^{\infty} \ln{\left( 1+\frac{1}{2^u} \right)} \leqslant \sum_{k=1}^{2s+1} (-1)^{k-1} \frac{1}{k(2^k-1)} \]
而级数
\[ \sum_{n=1}^{\infty} (-1)^{n-1} \frac{1}{n(2^n-1)} \]
显然是收敛的,因此在前式中两侧,分别令$r \to \infty$和$s \to \infty$便得
\begin{statement}
\[ \sum_{n=1}^{\infty} \ln{\left( 1+\frac{1}{2^n} \right)} = \sum_{n=1}^{\infty} (-1)^{n-1} \frac{1}{n(2^n-1)} \]
\end{statement}
不知这个级数的和是否有简单表示。

上述结果还有另一种导出法,利用$\ln{(1+x)}$的泰勒级数展开得
\[ \sum_{n=1}^{\infty} \ln{\left( 1+\frac{1}{2^n} \right)} = \sum_{n=1}^{\infty} \sum_{m=1}^{\infty} (-1)^{m-1} \frac{1}{m2^{nm}} \]
因为
\[ \frac{1}{m2^{nm}} < \frac{1}{2^{nm}} \]
所以前式右边的累级数是绝对收敛的,于是可以交换求和顺序,所以
\[ \sum_{n=1}^{\infty} \ln{\left( 1+\frac{1}{2^n} \right)} =  \sum_{m=1}^{\infty} \sum_{n=1}^{\infty}(-1)^{m-1} \frac{1}{m2^{nm}} = \sum_{m=1}^{\infty} (-1)^{m-1}\frac{1}{m(2^m-1)} \]

\section{关于江苏卷数列题目的讨论}
\label{sec:jiangshu-series-extend}

\textbf{作者}: 成都-zhcosin, 2017-06-10

  设数列是$P(m)$数列,则有递推公式
  \[ a_{n-m}+a_{n-m+1}+\cdots+a_{n-1}-2ma_n+a_{n+1}+\cdots+a_{n-m+1}+a_{n+m}=0 \]
  这是齐次线性递推数列,其特征方程是
  \[ f_m(x) = 1+x+\cdots+x^{m-1}-2mx^m+x^{m+1}+\cdots+x^{2m}=0 \]
  为了得到它的通项表示,讨论它的根的情况,因为$f_m(1)=f_m'(1)=0$,所以$x=1$是它的二重根。设它的其它复数根分别为$t_i(i=2,3,\cdots,q)$,则通项可表为
  \[ a_n=c_0+c_1 n + \sum_{i=2}^q p_i(n)t_i^n \]
  其中的系数$c_0,c_1$为复常数,$p_i(n)$是一些关于$n$的多项式。

  要使它是等差数列,则只需要通项中那些多项式全是零多项式就行了。如果数列既是$P(r)$数列,又是$P(s)$数列,则它的通项将有两种上述表示,本文的结论便是
  \begin{statement}
    如果某数列既是$P(r)$数列又是$P(s)$数列,且它俩的特征方程左边的多项式$f_r(x)$与$f_s(x)$除二重根$x=1$以外没有其它相同的根,则该数列是等差数列。
  \end{statement}
  有这结论之后,$f_2(x)$除二重根$x=1$以外还有两个实根(四次方程),而$f_3(x)$除二重根$x=1$以外还有四个虚根(六次方程),因此由这结论,便知原题结论正确。

  为了证明这个命题,只需要证明如下的命题:
\begin{statement}
  \label{statement:p-t-0}
    设$t_1,t_2,\cdots,t_m$是$m$个非零且互不相等的复常数,而$p_i(x)(i=1,2,\cdots,m)$一些关于$x$的多项式,如果等式$p_1(n)t_1^n+p_2(n)t_2^n+\cdots+p_m(n)t_m^n=0$对一切正整数$n$都成立,则$p_i(x)(i=1,2,\ldots,n)$全是零多项式。
\end{statement}

下面来一步步的证明它,首先容易证明
\begin{lemma}
  \label{lemma:ration-equation-bounded}
  对于任意两个非零复系数多项式$p(x)$和$q(x)$,存在正整数$n$,使得当$|x|$充分大时有
  \[ \left| \frac{p(x)}{q(x)} \right| < |x|^n \]
\end{lemma}

下面来证明命题\ref{statement:p-t-0},
\begin{proof}[\textbf{证明}]
  设在诸$t_i(i=1,2,\ldots,m)$中绝对值最大者为$t_m$,在原式两边同除以它的系数$p_m(n)$得
  \[ t_m^n = q_1(n)t_1^n+q_2(n)t_2^n+\cdots+q_{m-1}t_{m-1}^n \]
  其中
  \[ q_i(x) = - \frac{p_i(x)}{p_m(x)} (i=1,2,\ldots,m-1) \]
  由引理\ref{lemma:ration-equation-bounded},存在正整数$u$,使得当$n$充分大时,
  \[ |t_m^n|=|q_1(n)t_1^n+q_2(n)t_2^n+\cdots+q_{m-1}t_{m-1}^n| < n^u(|t_1|^n+|t_2|^n+\cdots+|t_{m-1}|^n|) \]
  于是令$r_i=t_i/t_m$,则$|r_i| < 1$,于是
  \[ n^u(|r_1|^n+|r_2|^n+\cdots+|r_{m-1}|^n) > 1  \]
  而显然这是不可能对一切正整数都成立的,因为左边极限为0.
\end{proof}

遗留问题:
1. 在什么情况下$f_r(x)$与$f_s(x)$除1以外没有其它相同的根.
2. 是否有$f_m(x)$与$f_{m+1}(x)$就符合条件.
 
\end{document}

%%% Local Variables:
%%% mode: latex
%%% TeX-master: t
%%% End:
